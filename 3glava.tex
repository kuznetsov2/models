\section{Тема 4. Игры в чистых стратегиях}

Антагонистические игры, в которых каждый игрок имеет конечное множество стратегий, называются матричными играми.

Итак, матричная игра --- это конечная игра двух лиц с нулевой суммой (т. е. сумма выигрышей игроков в каждой ситуации равна нулю). Такая игра полностью определяется матрицей
\[ H = \begin{pmatrix}
h_{11} & h_{12} & \dots & h_{1n} \\ 
h_{21} & h_{22} & \dots & h_{2n} \\ 
\dots & \dots & \dots & \dots \\ 
h_{m1} & h_{m2} & \dots & h_{mn}
\end{pmatrix},  \]
в которой строки соответствуют чистым стратегиям игрока 1, столбцы --- чистым стратегиям игрока 2, на их пересечении стоит выигрыш игрока 1 в соответствующей ситуации, т. е. ситуации $s = (i, j)$ соответствует выигрыш $H_1(s) \equiv H (i, j) =  h_{ij}$. Тогда выигрыш игрока 2 равен $H_2(s) = - H_1(s)$ для всех $s \in S$.

Здесь игрок 1 имеет $m$ стратегий, игрок 2 имеет $n$ стратегий. Такая игра называется $m\times n$‑игрой. Матрица $H$ называется матрицей игры или матрицей выигрышей (платежной матрицей).

Цель игрока 1 --- максимизировать cвой возможный выигрыш, при этом увеличение его выигрыша ведет к уменьшению выигрыша игрока 2 (так как игра антагонистическая). Аналогичное можно отметить и для игрока 2: увеличение его выигрыша ведет к уменьшению выигрыша игрока 1. Поэтому при выборе стратегии игрок 1 (разумный игрок, действующий рационально) будет руководствоваться следующими соображениями. При стратегии $i$ игрока 1 игрок 2 выберет стратегию $j_*$, максимизирующую его (игрока 2) выигрыш (тем самым минимизирующую выигрыш игрока 1):
\[ h_{ij*} = \min_j h_{ij}.\]

Тогда оптимальная стратегия игрока 1, которая обеспечит ему наибольший из возможных выигрышей $h_{ij*}, i = 1, 2, ..., m$, (т. е. при любой стратегии игрока 2), будет состоять в выборе стратегии $i_*$, для которой выполняется:
\[ h_{i*j*} = \max_i h_{ij*} = \max_i \min_j h_{ij}. \]

Аналогичными соображениями будет руководствоваться игрок 2 при выборе стратегии: обеспечить наибольший возможный выигрыш при любом выборе стратегии игрока 1, т. е. выбрать стратегию, которая обеспечит ему $max$ из возможных выигрышей
\[ -h_{i*j},j = 1, 2,..., n, \ \text{здесь} \ h_{i*j} = \max_i h_{ij}, \]
причем для второго игрока выигрыш равен $–h$, где $h$ --- выигрыш игрока 1.

Таким образом, оптимальная стратегия игрока 2 будет состоять в выборе стратегии $j*$, для которой выполняется:
\[ -h _{i*j*} = \max_j (-h_{i*j}) = \max_j (-\max_i h_{ij}) = - \min_j \max_i h_{ij}, \]
отсюда получим:
\[h_{i*j*} = \min_j \max_i h_{ij}.\]