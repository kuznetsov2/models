\section{Тема 1. Устойчивость оптимизационного решения}

После того как оптимальное решение задачи найдено, проводится его экономико-математический анализ. Одним из его наиболее важных элементов является изучение того, как влияют на решение изменения различных параметров модели.

Такое исследование называется анализом устойчивости решения. Оно позволяет выяснить, насколько решение модели чувствительно к изменению внешних условий, а также определить область изменения параметров, в которой оно остается прежним.

Проиллюстрируем роль анализа устойчивости на примере задачи фирмы. Ее решение является наилучшим планом выпуска в конкретной экономической ситуации, которая характеризуется определенными условиями производства и сбыта продукции.

Они находят свое отражение в математической модели в виде фиксированных значений ее параметров: удельной прибыли изделий, наличных объемов ресурсов и нормативов их затрат.

При принятии решения ЛПР необходимо учитывать по крайней мере два обстоятельства: во-первых, значения параметров модели обычно известны неточно; во-вторых, часть из них, например прибыльность изделий, может измениться уже в ближайшем будущем

Поэтому если окажется, что небольшие изменения параметров сильно влияют на характеристики оптимального плана, то его реализация без дополнительного изучения модели не представляется разумной.

Это изучение должно включать уточнение значений параметров и области их вероятных изменений и может привести к заключению о необходимости корректировки самой модели.

Если же выяснится, что возможные колебания параметров мало или вообще не влияют на найденное решение, то его выбор в качестве плана производства будет более обоснованным.

Таким образом, анализ устойчивости должен предшествовать использованию результатов расчетов по модели при принятии управленческих решений.

Отчет по устойчивости содержит основную информацию для анализа устойчивости решения. Он состоит из двух таблиц (рис. ).

% Please add the following required packages to your document preamble:
% \usepackage{graphicx}
\begin{table}[!h]
	\caption{Отчет по устойчивости}
	\small
	\setlength\extrarowheight{5pt}
	\resizebox{\textwidth}{!}{%
		\begin{tabular}{|l|l|l|l|l|l|l|}
			\hline
			\multicolumn{7}{|l|}{\textbf{Изменяемые ячейки}}                                                                                                   \\ \hline
			\textbf{Ячейка} & \textbf{Имя}        & \textbf{Результ.} & \textbf{Нормир.}   & \textbf{Целевой}      & \textbf{Допустимое} & \textbf{Допустимое} \\
			&                     & \textbf{значение} & \textbf{стоимость} & \textbf{Коэффициент}  & \textbf{Увеличение} & \textbf{Уменьшение} \\ \hline
			\$B\$10         & Выпуск Изделие 1    & 56                & 0                  & 25                    & 1,1538              & 2,142 857           \\ \hline
			\$C\$10         & Выпуск Изделие 2    & 18                & 0                  & 40                    & 3,75                & 0,9375              \\ \hline
			\$D\$10         & Выпуск Изделие 3    & 0                 & -1,5               & 30                    & 1,5                 & 1E+30               \\ \hline
			\multicolumn{7}{|l|}{\textbf{Ограничения}}                                                                                                         \\ \hline
			\textbf{Ячейка} & \textbf{Имя}        & \textbf{Результ.} & \textbf{Теневая}   & \textbf{Ограничение}  & \textbf{Допустимое} & \textbf{Допустимое} \\
			&                     & \textbf{значение} & \textbf{цена}      & \textbf{Правая часть} & \textbf{Увеличение} & \textbf{Уменьшение} \\ \hline
			\$E\$5          & Сырье Расход        & 388               & 0                  & 400                   & 1E+30               & 12                  \\ \hline
			\$E\$6          & Оборудование Расход & 350               & 4                  & 350                   & 70                  & 30                  \\ \hline
			\$E\$7          & Труд Расход         & 480               & 1,5                & 480                   & 10,909              & 80                  \\ \hline
		\end{tabular}%
	}
\end{table}

Первая таблица <<Изменяемые ячейки>> содержит сведения о чувствительности оптимального решения и оптимального значения целевой функции к малым изменениям ее коэффициентов. Ее столбцы содержат следующую информацию:

<<Резулът. значение>> --- оптимальные значения переменных (объемы выпуска).

<<Нормир. стоимость>> --- двойственные оценки переменных, которые показывают, насколько изменится оптимальное значение целевой функ­ции, если принудительно включить единицу изделия этого вида в оптимальный план.

Эта оценка отлична от нуля лишь для изделий, не вошедших в оптимальный план. Так, оценка изделия 3 равна -1,5. Это означает, что если установить фирме обязательное задание по выпуску единицы этого изделия, т. е. заменить условие $x_3 \geq 0$ на $x_3 \geq 1$, то оптимальное значение прибыли уменьшится на 1,5 и составит 2118,5.

<<Целевой Коэффициент>> --- коэффициенты целевой функции (удельная прибыль изделия).

<<Допустимое Увеличение (Уменьшение)>> --- насколько можно увеличить (уменьшить) соответствующий коэффициент целевой функции (удельную прибыль изделия), чтобы оптимальное решение не изменилось.

Таким образом, при изменении первого коэффициента целевой функции (удельной прибыли изделия 1) в интервале $I_1= (22,86; 26,15)$ решение останется прежним.

Поэтому этот интервал называют интервалом устойчивости решения. Если же значение удельной прибыли выйдет за его пределы, то это приведет к изменению оптимального плана.% (см. ниже табл. 2.2).

Интервалы устойчивости для остальных коэффициентов целевой функции таковы:

\[ I_2 = (39, 06 ; 43,75)\ \text{и} \  I_3 = ( -\infty ;31,5). \]

Во второй таблице <<Ограничения>> находится информация об оптимальных оценках ограничений. Ее столбцы содержат следующие сведения:

<<Результ. значение>> --- значение левой части ограничения (затраты ресурсов) в оптимальном плане.

<<Теневая Цена>> --- двойственные оценки ограничений (ресурсов), показывающие, насколько изменится оптимальное значение целевой функции, если увеличить на единицу правую часть ограничения (наличный объем ресурса).

Таким образом, ресурсы имеют следующие оценки: сырье -0, оборудование -4 и труд -1,5. Это означает, что дополнительная единица сырья не приведет к увеличению прибыли фирмы, дополнительная единица оборудования позволит фирме увеличить свою прибыль на 4 единицы, а труда --- на 1,5 единицы



<<Ограничение Правая часть>> --- значения правых частей ограничений (наличные объемы ресурсов).

<<Допустимое Увеличение (Уменьшение)>> --- насколько можно увеличить (уменьшить) правую часть соответствующего ограничения, чтобы не изменилась его двойственная оценка (теневая цена).

Информация, содержащаяся в последних трех столбцах, позволяет найти интервалы устойчивости оценок, в пределах которых их значения не изменяются. Левая граница интервала вычисляется по формуле

<<Ограничение Правая часть>> -- <<Допустимое Уменьшение>>, а правая граница --- по формуле:

<<Ограничение Правая часть>> + <<Допустимое Увеличение>>.

Таким образом, интервал устойчивости оценки сырья имеет вид $J_1 = (388; +\infty)$. В нем оценка равна 0, так как этот ресурс избыточен. Ин­тервал устойчивости оценки оборудования $J_2 = (320; 420)$.

В его пределах каждая дополнительная единица оборудования позволяет фирме увеличить прибыль на 4 единицы. Соответственно, интервал устойчиво­сти оценки труда $J_3 = (400; 490,9)$.