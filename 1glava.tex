\section{Тема 2. Трендовые модели на основе кривых роста}

Использование метода экстраполяции на основе кривых роста для прогнозирования базируется на двух предположениях:
\begin{itemize}
	\item временной ряд экономического показателя действительно имеет тренд, т.е. преобладающую тенденцию;
	\item общие условия, определявшие развитие показателя в прошлом, останутся без существенных изменений в течение периода упреждения.
\end{itemize}

В настоящее время насчитывается большое количество типов кривых роста для экономических процессов. Чтобы правильно подобрать наилучшую кривую роста для моделирования и прогнозирования экономического явления, необходимо знать особенности каждого вида кривых. Наиболее часто в экономике используются полиномиальные, экспоненциальные и S-образные кривые роста. Простейшие полиномиальные кривые роста имеют вид:

\[ \hat{y_t} = a_0 + a_1t \ \text{(полином первой степени)}\]
\[ \hat{y_t} = a_0 + a_1t +a_2t^2 \ \text{(полином второй степени)}\]
\[ \hat{y_t} = a_0 + a_1t +a_2t^2 + a_3t^3 \ \text{(полином третьей степени)}\]
и т.д.

Параметр $а_1$ называют линейным приростом, параметр $а_2$ --- ускорением роста, параметр $а_3$ --- изменением ускорения роста.

Для полинома первой степени характерен постоянный закон роста. Если рассчитать первые приросты по формуле $u_t = y_t - y_{t-1}, t = 2, 3, ..., n$, то они будут постоянной величиной и равны $a_1$.

Если первые приросты рассчитать для полинома второй степени, то они будут иметь линейную зависимость от времени и ряд из первых приростов $u_2, u_3, ...$ на графике будет представлен прямой линией. Вторые приросты $u_t^{(2)} = u_t - u_{t-1}$ для полинома второй степени будут постоянны.

Для полинома третьей степени первые приросты будут полиномами второй степени, вторые приросты будут линейной функцией времени, а третьи приросты, рассчитываемые по формуле $u_t^{(3)} = u_t^{(2)} - u_{t-1}^{(2)}$, будут постоянной величиной.

На основе сказанного можно отметить следующие свойства полиномиальных кривых роста:
\begin{itemize}
	\item от полинома высокого порядка можно путем расчета последовательных разностей (приростов) перейти к полиному более низкого порядка;
	\item значения приростов для полиномов любого порядка не зависят от значений самой функции $\hat{y_t}$.
\end{itemize}

Таким образом, полиномиальные кривые роста можно использовать для аппроксимации (приближения) и прогнозирования экономических процессов, в которых последующее развитие не зависит от достигнутого уровня.

В отличие от использования полиномиальных кривых использование экспоненциальных кривых роста предполагает, что дальнейшее развитие зависит от достигнутого уровня, например, прирост зависит от значения функции. В экономике чаще всего применяются две разновидности экспоненциальных (показательных) кривых: простая экспонента и модифицированная экспонента.




























